%% Generated by Sphinx.
\def\sphinxdocclass{report}
\documentclass[letterpaper,10pt,spanish]{sphinxmanual}
\ifdefined\pdfpxdimen
   \let\sphinxpxdimen\pdfpxdimen\else\newdimen\sphinxpxdimen
\fi \sphinxpxdimen=.75bp\relax
\ifdefined\pdfimageresolution
    \pdfimageresolution= \numexpr \dimexpr1in\relax/\sphinxpxdimen\relax
\fi
%% let collapsible pdf bookmarks panel have high depth per default
\PassOptionsToPackage{bookmarksdepth=5}{hyperref}

\PassOptionsToPackage{booktabs}{sphinx}
\PassOptionsToPackage{colorrows}{sphinx}

\PassOptionsToPackage{warn}{textcomp}
\usepackage[utf8]{inputenc}
\ifdefined\DeclareUnicodeCharacter
% support both utf8 and utf8x syntaxes
  \ifdefined\DeclareUnicodeCharacterAsOptional
    \def\sphinxDUC#1{\DeclareUnicodeCharacter{"#1}}
  \else
    \let\sphinxDUC\DeclareUnicodeCharacter
  \fi
  \sphinxDUC{00A0}{\nobreakspace}
  \sphinxDUC{2500}{\sphinxunichar{2500}}
  \sphinxDUC{2502}{\sphinxunichar{2502}}
  \sphinxDUC{2514}{\sphinxunichar{2514}}
  \sphinxDUC{251C}{\sphinxunichar{251C}}
  \sphinxDUC{2572}{\textbackslash}
\fi
\usepackage{cmap}
\usepackage[T1]{fontenc}
\usepackage{amsmath,amssymb,amstext}
\usepackage{babel}



\usepackage{tgtermes}
\usepackage{tgheros}
\renewcommand{\ttdefault}{txtt}



\usepackage[Sonny]{fncychap}
\ChNameVar{\Large\normalfont\sffamily}
\ChTitleVar{\Large\normalfont\sffamily}
\usepackage{sphinx}

\fvset{fontsize=auto}
\usepackage{geometry}


% Include hyperref last.
\usepackage{hyperref}
% Fix anchor placement for figures with captions.
\usepackage{hypcap}% it must be loaded after hyperref.
% Set up styles of URL: it should be placed after hyperref.
\urlstyle{same}

\addto\captionsspanish{\renewcommand{\contentsname}{Contents:}}

\usepackage{sphinxmessages}
\setcounter{tocdepth}{1}



\title{Proyecto:\_PIA\_AA}
\date{02 de mayo de 2025}
\release{1.0}
\author{Noel Freire Mahía 
 Iván Hermida Mella}
\newcommand{\sphinxlogo}{\vbox{}}
\renewcommand{\releasename}{Versión}
\makeindex
\begin{document}

\ifdefined\shorthandoff
  \ifnum\catcode`\=\string=\active\shorthandoff{=}\fi
  \ifnum\catcode`\"=\active\shorthandoff{"}\fi
\fi

\pagestyle{empty}
\sphinxmaketitle
\pagestyle{plain}
\sphinxtableofcontents
\pagestyle{normal}
\phantomsection\label{\detokenize{index::doc}}


\sphinxAtStartPar
Add your content using \sphinxcode{\sphinxupquote{reStructuredText}} syntax. See the
\sphinxhref{https://www.sphinx-doc.org/en/master/usage/restructuredtext/index.html}{reStructuredText}
documentation for details.

\sphinxstepscope


\chapter{Funciones Generales}
\label{\detokenize{Funciones_Generales:module-subir_datos}}\label{\detokenize{Funciones_Generales:funciones-generales}}\label{\detokenize{Funciones_Generales::doc}}\index{module@\spxentry{module}!subir\_datos@\spxentry{subir\_datos}}\index{subir\_datos@\spxentry{subir\_datos}!module@\spxentry{module}}\index{cargar\_datos() (en el módulo subir\_datos)@\spxentry{cargar\_datos()}\spxextra{en el módulo subir\_datos}}

\begin{fulllineitems}
\phantomsection\label{\detokenize{Funciones_Generales:subir_datos.cargar_datos}}
\pysigstartsignatures
\pysiglinewithargsret
{\sphinxcode{\sphinxupquote{subir\_datos.}}\sphinxbfcode{\sphinxupquote{cargar\_datos}}}
{\sphinxparam{\DUrole{n}{path}}\sphinxparamcomma \sphinxparam{\DUrole{n}{target}}}
{}
\pysigstopsignatures
\sphinxAtStartPar
Carga los datos de un archivo CSV.
\begin{description}
\sphinxlineitem{Args:}
\sphinxAtStartPar
path (str): Ruta de los datos a normalizar.
target (str): Nombre de la columna que contiene las etiquetas.

\sphinxlineitem{Returns:}
\sphinxAtStartPar
X (Dataframe): Datos que se enviarán al modelo.
y (Series): Etiquetas de cada dato.

\end{description}

\end{fulllineitems}

\index{module@\spxentry{module}!normalizar@\spxentry{normalizar}}\index{normalizar@\spxentry{normalizar}!module@\spxentry{module}}\index{normalizar\_datos() (en el módulo normalizar)@\spxentry{normalizar\_datos()}\spxextra{en el módulo normalizar}}\phantomsection\label{\detokenize{Funciones_Generales:module-normalizar}}

\begin{fulllineitems}
\phantomsection\label{\detokenize{Funciones_Generales:normalizar.normalizar_datos}}
\pysigstartsignatures
\pysiglinewithargsret
{\sphinxcode{\sphinxupquote{normalizar.}}\sphinxbfcode{\sphinxupquote{normalizar\_datos}}}
{\sphinxparam{\DUrole{n}{x}}}
{}
\pysigstopsignatures
\sphinxAtStartPar
Normaliza los datos utilizando StandardScaler (media 0, desviación típica 1) de Scikit\sphinxhyphen{}Learn.
\begin{description}
\sphinxlineitem{Args:}
\sphinxAtStartPar
x (Dataframe): Datos a normalizar.

\sphinxlineitem{Returns:}
\sphinxAtStartPar
x\_scaled (Dataframe): Datos normalizados.
scalers (dict): Diccionario de scalers utilizados para cada columna.

\end{description}

\end{fulllineitems}

\index{module@\spxentry{module}!guardar\_metricas@\spxentry{guardar\_metricas}}\index{guardar\_metricas@\spxentry{guardar\_metricas}!module@\spxentry{module}}\index{guardar\_metricas() (en el módulo guardar\_metricas)@\spxentry{guardar\_metricas()}\spxextra{en el módulo guardar\_metricas}}\phantomsection\label{\detokenize{Funciones_Generales:module-guardar_metricas}}

\begin{fulllineitems}
\phantomsection\label{\detokenize{Funciones_Generales:guardar_metricas.guardar_metricas}}
\pysigstartsignatures
\pysiglinewithargsret
{\sphinxcode{\sphinxupquote{guardar\_metricas.}}\sphinxbfcode{\sphinxupquote{guardar\_metricas}}}
{\sphinxparam{\DUrole{n}{k}}\sphinxparamcomma \sphinxparam{\DUrole{n}{metrics}}\sphinxparamcomma \sphinxparam{\DUrole{n}{file}}\sphinxparamcomma \sphinxparam{\DUrole{n}{histories}}\sphinxparamcomma \sphinxparam{\DUrole{n}{roc}}\sphinxparamcomma \sphinxparam{\DUrole{n}{n\_jobs}}}
{}
\pysigstopsignatures
\sphinxAtStartPar
Guarda las métricas del modelo en un archivo Excel, incluyendo información sobre las predicciones, 
evolución del entrenamiento y curvas ROC si están disponibles.

\sphinxAtStartPar
El archivo Excel generado puede contener varias hojas:
\sphinxhyphen{} \sphinxtitleref{k\_metrics}: Métricas generales del modelo por cada fold de la validación cruzada.
\sphinxhyphen{} \sphinxtitleref{y\_test}: Clases reales de las imágenes utilizadas en la validación.
\sphinxhyphen{} \sphinxtitleref{y\_pred}: Clases predichas por el modelo.
\sphinxhyphen{} \sphinxtitleref{histories} (opcional): Evolución del accuracy y la pérdida durante el entrenamiento.
\sphinxhyphen{} \sphinxtitleref{roc} (opcional): Datos para la curva ROC global y por clases.
\begin{quote}\begin{description}
\sphinxlineitem{Parámetros}\begin{itemize}
\item {} 
\sphinxAtStartPar
\sphinxstyleliteralstrong{\sphinxupquote{k}} (\sphinxstyleliteralemphasis{\sphinxupquote{int}}) \textendash{} Número de folds en la validación cruzada.

\item {} 
\sphinxAtStartPar
\sphinxstyleliteralstrong{\sphinxupquote{metrics}} (\sphinxstyleliteralemphasis{\sphinxupquote{dict}}) \textendash{} Diccionario con las métricas del modelo, incluyendo las predicciones y opcionalmente 
datos de entrenamiento y curvas ROC.

\item {} 
\sphinxAtStartPar
\sphinxstyleliteralstrong{\sphinxupquote{file}} (\sphinxstyleliteralemphasis{\sphinxupquote{str}}) \textendash{} Nombre del archivo Excel donde se guardarán las métricas.

\item {} 
\sphinxAtStartPar
\sphinxstyleliteralstrong{\sphinxupquote{histories}} (\sphinxstyleliteralemphasis{\sphinxupquote{bool}}) \textendash{} Indica si \sphinxtitleref{metrics} contiene los datos de evolución del accuracy y pérdida.

\item {} 
\sphinxAtStartPar
\sphinxstyleliteralstrong{\sphinxupquote{roc}} (\sphinxstyleliteralemphasis{\sphinxupquote{bool}}) \textendash{} Indica si \sphinxtitleref{metrics} contiene los datos para generar curvas ROC.

\end{itemize}

\sphinxlineitem{Devuelve}
\sphinxAtStartPar
No retorna ningún valor, solo genera y guarda un archivo Excel con las métricas.

\sphinxlineitem{Tipo del valor devuelto}
\sphinxAtStartPar
None

\end{description}\end{quote}

\end{fulllineitems}

\index{module@\spxentry{module}!metricas\_clustering@\spxentry{metricas\_clustering}}\index{metricas\_clustering@\spxentry{metricas\_clustering}!module@\spxentry{module}}\index{guardar\_metricas\_clustering() (en el módulo metricas\_clustering)@\spxentry{guardar\_metricas\_clustering()}\spxextra{en el módulo metricas\_clustering}}\phantomsection\label{\detokenize{Funciones_Generales:module-metricas_clustering}}

\begin{fulllineitems}
\phantomsection\label{\detokenize{Funciones_Generales:metricas_clustering.guardar_metricas_clustering}}
\pysigstartsignatures
\pysiglinewithargsret
{\sphinxcode{\sphinxupquote{metricas\_clustering.}}\sphinxbfcode{\sphinxupquote{guardar\_metricas\_clustering}}}
{\sphinxparam{\DUrole{n}{file}}\sphinxparamcomma \sphinxparam{\DUrole{n}{metrics}}\sphinxparamcomma \sphinxparam{\DUrole{n}{etiquetas\_real}}\sphinxparamcomma \sphinxparam{\DUrole{n}{etiquetas\_pred}}\sphinxparamcomma \sphinxparam{\DUrole{n}{centroides}\DUrole{o}{=}\DUrole{default_value}{None}}}
{}
\pysigstopsignatures
\sphinxAtStartPar
Guarda las métricas de evaluación, etiquetas reales, etiquetas predichas y, si están disponibles, 
los centroides del clustering en un archivo Excel.
\begin{description}
\sphinxlineitem{Args:}
\sphinxAtStartPar
file (str): Ruta y nombre del archivo Excel donde se guardarán los resultados.
metrics (dict): Diccionario con métricas de evaluación (V\sphinxhyphen{}Measure e Índice de Rand ajustado).
etiquetas\_real (array\sphinxhyphen{}like): Etiquetas reales de los datos.
etiquetas\_pred (array\sphinxhyphen{}like): Etiquetas asignadas por el modelo de clustering.
centroides (ndarray, optional): Coordenadas de los centroides de los clústers (si aplica).

\sphinxlineitem{Returns:}
\sphinxAtStartPar
None: Los datos se guardan en el archivo Excel especificado.

\end{description}

\end{fulllineitems}

\index{module@\spxentry{module}!comparacion\_n\_covariables@\spxentry{comparacion\_n\_covariables}}\index{comparacion\_n\_covariables@\spxentry{comparacion\_n\_covariables}!module@\spxentry{module}}\index{comparar\_todos\_modelos() (en el módulo comparacion\_n\_covariables)@\spxentry{comparar\_todos\_modelos()}\spxextra{en el módulo comparacion\_n\_covariables}}\phantomsection\label{\detokenize{Funciones_Generales:module-comparacion_n_covariables}}

\begin{fulllineitems}
\phantomsection\label{\detokenize{Funciones_Generales:comparacion_n_covariables.comparar_todos_modelos}}
\pysigstartsignatures
\pysiglinewithargsret
{\sphinxcode{\sphinxupquote{comparacion\_n\_covariables.}}\sphinxbfcode{\sphinxupquote{comparar\_todos\_modelos}}}
{\sphinxparam{\DUrole{n}{acc}}\sphinxparamcomma \sphinxparam{\DUrole{n}{modelo}}}
{}
\pysigstopsignatures
\sphinxAtStartPar
Compara todos los modelos usando la prueba de Kruskal\sphinxhyphen{}Wallis y la prueba de Tukey si hay diferencias significativas.
\begin{quote}\begin{description}
\sphinxlineitem{Parámetros}
\sphinxAtStartPar
\sphinxstyleliteralstrong{\sphinxupquote{acc}} (\sphinxstyleliteralemphasis{\sphinxupquote{dict}}) \textendash{} Diccionario con las precisiones de los modelos.

\sphinxlineitem{Devuelve}
\sphinxAtStartPar
El nombre del mejor modelo y su media de precisión.

\sphinxlineitem{Tipo del valor devuelto}
\sphinxAtStartPar
tuple(str, float)

\end{description}\end{quote}

\end{fulllineitems}

\index{diagrama\_cajas() (en el módulo comparacion\_n\_covariables)@\spxentry{diagrama\_cajas()}\spxextra{en el módulo comparacion\_n\_covariables}}

\begin{fulllineitems}
\phantomsection\label{\detokenize{Funciones_Generales:comparacion_n_covariables.diagrama_cajas}}
\pysigstartsignatures
\pysiglinewithargsret
{\sphinxcode{\sphinxupquote{comparacion\_n\_covariables.}}\sphinxbfcode{\sphinxupquote{diagrama\_cajas}}}
{\sphinxparam{\DUrole{n}{accuracy}}}
{}
\pysigstopsignatures
\sphinxAtStartPar
Genera un diagrama de cajas (boxplot) para comparar la precisión de diferentes modelos.
\begin{quote}\begin{description}
\sphinxlineitem{Parámetros}
\sphinxAtStartPar
\sphinxstyleliteralstrong{\sphinxupquote{accuracy}} (\sphinxstyleliteralemphasis{\sphinxupquote{dict}}) \textendash{} Diccionario que contiene las precisiones de los modelos.

\sphinxlineitem{Devuelve}
\sphinxAtStartPar
None

\end{description}\end{quote}

\end{fulllineitems}

\index{inicio\_comp() (en el módulo comparacion\_n\_covariables)@\spxentry{inicio\_comp()}\spxextra{en el módulo comparacion\_n\_covariables}}

\begin{fulllineitems}
\phantomsection\label{\detokenize{Funciones_Generales:comparacion_n_covariables.inicio_comp}}
\pysigstartsignatures
\pysiglinewithargsret
{\sphinxcode{\sphinxupquote{comparacion\_n\_covariables.}}\sphinxbfcode{\sphinxupquote{inicio\_comp}}}
{\sphinxparam{\DUrole{n}{Param\_default}}\sphinxparamcomma \sphinxparam{\DUrole{n}{ICA\_4}}\sphinxparamcomma \sphinxparam{\DUrole{n}{ICA\_8}}\sphinxparamcomma \sphinxparam{\DUrole{n}{ICA\_11}}\sphinxparamcomma \sphinxparam{\DUrole{n}{PCA\_4}}\sphinxparamcomma \sphinxparam{\DUrole{n}{PCA\_8}}\sphinxparamcomma \sphinxparam{\DUrole{n}{PCA\_11}}\sphinxparamcomma \sphinxparam{\DUrole{n}{modelo}}}
{}
\pysigstopsignatures
\sphinxAtStartPar
Función principal que inicia el proceso de comparación de modelos.

\sphinxAtStartPar
Lee los archivos de precisión, compara modelos y guarda los resultados.
\begin{quote}\begin{description}
\sphinxlineitem{Devuelve}
\sphinxAtStartPar
None

\end{description}\end{quote}

\end{fulllineitems}


\sphinxstepscope


\chapter{Funciones Clustering}
\label{\detokenize{Clustering:module-clustering}}\label{\detokenize{Clustering:funciones-clustering}}\label{\detokenize{Clustering::doc}}\index{module@\spxentry{module}!clustering@\spxentry{clustering}}\index{clustering@\spxentry{clustering}!module@\spxentry{module}}\index{dbscan() (en el módulo clustering)@\spxentry{dbscan()}\spxextra{en el módulo clustering}}

\begin{fulllineitems}
\phantomsection\label{\detokenize{Clustering:clustering.dbscan}}
\pysigstartsignatures
\pysiglinewithargsret
{\sphinxcode{\sphinxupquote{clustering.}}\sphinxbfcode{\sphinxupquote{dbscan}}}
{\sphinxparam{\DUrole{n}{x}}\sphinxparamcomma \sphinxparam{\DUrole{n}{y}}}
{}
\pysigstopsignatures
\sphinxAtStartPar
Aplica el algoritmo DBSCAN, calcula métricas 
de calidad del clustering y visualiza la distribución de distancias para 
seleccionar un buen valor de epsilon.
\begin{description}
\sphinxlineitem{Args:}
\sphinxAtStartPar
x (ndarray): Datos de entrada para el clustering.
y (array\sphinxhyphen{}like): Etiquetas reales de los datos (para evaluación).

\sphinxlineitem{Returns:}
\sphinxAtStartPar
None: Los resultados se guardan en un archivo Excel y se muestra una gráfica.

\end{description}

\end{fulllineitems}

\index{kmeans() (en el módulo clustering)@\spxentry{kmeans()}\spxextra{en el módulo clustering}}

\begin{fulllineitems}
\phantomsection\label{\detokenize{Clustering:clustering.kmeans}}
\pysigstartsignatures
\pysiglinewithargsret
{\sphinxcode{\sphinxupquote{clustering.}}\sphinxbfcode{\sphinxupquote{kmeans}}}
{\sphinxparam{\DUrole{n}{x}}\sphinxparamcomma \sphinxparam{\DUrole{n}{y}}}
{}
\pysigstopsignatures
\sphinxAtStartPar
Aplica el algoritmo k\sphinxhyphen{}Means para encontrar el número óptimo de clústers 
utilizando las métricas de inercia y coeficiente de silueta. Guarda los resultados 
y visualiza los clústers con t\sphinxhyphen{}SNE.
\begin{description}
\sphinxlineitem{Args:}
\sphinxAtStartPar
x (ndarray): Datos de entrada para el clustering.
y (array\sphinxhyphen{}like): Etiquetas reales de los datos (para evaluación).

\sphinxlineitem{Returns:}
\sphinxAtStartPar
None: Los resultados se guardan en un archivo Excel y se muestra una gráfica.

\end{description}

\end{fulllineitems}


\sphinxstepscope


\chapter{Funciones de Reducción de Dimensionalidad}
\label{\detokenize{Reduccion_Dimension:module-Modelos}}\label{\detokenize{Reduccion_Dimension:funciones-de-reduccion-de-dimensionalidad}}\label{\detokenize{Reduccion_Dimension::doc}}\index{module@\spxentry{module}!Modelos@\spxentry{Modelos}}\index{Modelos@\spxentry{Modelos}!module@\spxentry{module}}\index{calculo\_tiempos() (en el módulo Modelos)@\spxentry{calculo\_tiempos()}\spxextra{en el módulo Modelos}}

\begin{fulllineitems}
\phantomsection\label{\detokenize{Reduccion_Dimension:Modelos.calculo_tiempos}}
\pysigstartsignatures
\pysiglinewithargsret
{\sphinxcode{\sphinxupquote{Modelos.}}\sphinxbfcode{\sphinxupquote{calculo\_tiempos}}}
{\sphinxparam{\DUrole{n}{k}}\sphinxparamcomma \sphinxparam{\DUrole{n}{X\_scaled}}\sphinxparamcomma \sphinxparam{\DUrole{n}{y}}\sphinxparamcomma \sphinxparam{\DUrole{n}{n\_componentes}}\sphinxparamcomma \sphinxparam{\DUrole{n}{reductor}}\sphinxparamcomma \sphinxparam{\DUrole{n}{modelo}}}
{}
\pysigstopsignatures
\sphinxAtStartPar
Realiza validación cruzada estratificada con un modelo XGBoost y calcula métricas de rendimiento.
\begin{description}
\sphinxlineitem{Args:}
\sphinxAtStartPar
k (int): Número de particiones para la validación cruzada.
X\_scaled(Dataframe): Lista de características extraídas de las imágenes.
y (series): Lista con las etiquetas de las imágenes.
n\_componentes (int): Número de componentes principales a utilizar.
reductor (str): Método de reducción de dimensionalidad (“PCA”, “ICA” o ninguno).
modelo (str): Tipo de modelo a utilizar (“rfc” o “knn”).

\sphinxlineitem{Returns:}
\sphinxAtStartPar
K\_metrics (dict): Diccionario con las métricas de rendimiento.

\end{description}

\end{fulllineitems}

\index{metricas\_ann() (en el módulo Modelos)@\spxentry{metricas\_ann()}\spxextra{en el módulo Modelos}}

\begin{fulllineitems}
\phantomsection\label{\detokenize{Reduccion_Dimension:Modelos.metricas_ann}}
\pysigstartsignatures
\pysiglinewithargsret
{\sphinxcode{\sphinxupquote{Modelos.}}\sphinxbfcode{\sphinxupquote{metricas\_ann}}}
{\sphinxparam{\DUrole{n}{kfold}}\sphinxparamcomma \sphinxparam{\DUrole{n}{X\_scaled}}\sphinxparamcomma \sphinxparam{\DUrole{n}{y}}\sphinxparamcomma \sphinxparam{\DUrole{n}{tiempo\_secuencial}}\sphinxparamcomma \sphinxparam{\DUrole{n}{tiempo\_multihilo}}\sphinxparamcomma \sphinxparam{\DUrole{n}{tiempo\_multiproceso}}}
{}
\pysigstopsignatures
\sphinxAtStartPar
Realiza validación cruzada estratificada con un modelo ANN y calcula métricas de rendimiento.
\begin{description}
\sphinxlineitem{Args:}
\sphinxAtStartPar
kfold (StratifiedKFold): Objeto de validación cruzada estratificada.
X\_scaled (DataFrame): Matriz de datos escalados que serán la entrada al modelo.
y (Series): Vector de las variables de salida del modelo.
tiempo\_secuencial (float): Tiempo de ejecución secuencial.
tiempo\_multihilo (float): Tiempo de ejecución multihilo.
tiempo\_multiproceso (float): Tiempo de ejecución multiproceso.

\sphinxlineitem{Returns:}
\sphinxAtStartPar
K\_metrics (dict): Diccionario con las métricas de rendimiento.

\end{description}

\end{fulllineitems}

\index{metricas\_rfc\_knn() (en el módulo Modelos)@\spxentry{metricas\_rfc\_knn()}\spxextra{en el módulo Modelos}}

\begin{fulllineitems}
\phantomsection\label{\detokenize{Reduccion_Dimension:Modelos.metricas_rfc_knn}}
\pysigstartsignatures
\pysiglinewithargsret
{\sphinxcode{\sphinxupquote{Modelos.}}\sphinxbfcode{\sphinxupquote{metricas\_rfc\_knn}}}
{\sphinxparam{\DUrole{n}{kfold}}\sphinxparamcomma \sphinxparam{\DUrole{n}{X\_scaled}}\sphinxparamcomma \sphinxparam{\DUrole{n}{y}}\sphinxparamcomma \sphinxparam{\DUrole{n}{tiempo\_secuencial}}\sphinxparamcomma \sphinxparam{\DUrole{n}{tiempo\_multihilo}}\sphinxparamcomma \sphinxparam{\DUrole{n}{tiempo\_multiproceso}}\sphinxparamcomma \sphinxparam{\DUrole{n}{tiempo\_n\_jobs}}\sphinxparamcomma \sphinxparam{\DUrole{n}{metodo}}}
{}
\pysigstopsignatures
\sphinxAtStartPar
Realiza validación cruzada estratificada con un modelo RFC o KNN y calcula métricas de rendimiento.
\begin{description}
\sphinxlineitem{Args:}
\sphinxAtStartPar
kfold (StratifiedKFold): Objeto de validación cruzada estratificada.
X\_scaled (DataFrame): Matriz de datos escalados que serán la entrada al modelo.
y (Series): Vector de las variables de salida del modelo.
tiempo\_secuencial (float): Tiempo de ejecución secuencial.
tiempo\_multihilo (float): Tiempo de ejecución multihilo.
tiempo\_multiproceso (float): Tiempo de ejecución multiproceso.
tiempo\_n\_jobs (float): Tiempo de ejecución con n\_jobs.
metodo (str): Método a utilizar (“RFC” o “KNN”).

\sphinxlineitem{Returns:}
\sphinxAtStartPar
K\_metrics (dict): Diccionario con las métricas de rendimiento.

\end{description}

\end{fulllineitems}

\index{model() (en el módulo Modelos)@\spxentry{model()}\spxextra{en el módulo Modelos}}

\begin{fulllineitems}
\phantomsection\label{\detokenize{Reduccion_Dimension:Modelos.model}}
\pysigstartsignatures
\pysiglinewithargsret
{\sphinxcode{\sphinxupquote{Modelos.}}\sphinxbfcode{\sphinxupquote{model}}}
{\sphinxparam{\DUrole{n}{X\_scaled}}\sphinxparamcomma \sphinxparam{\DUrole{n}{y}}\sphinxparamcomma \sphinxparam{\DUrole{n}{n\_componentes}}\sphinxparamcomma \sphinxparam{\DUrole{n}{reductor}}\sphinxparamcomma \sphinxparam{\DUrole{n}{modelo}}}
{}
\pysigstopsignatures
\sphinxAtStartPar
Entrena y evalúa un modelo XGBoost utilizando características escaladas.
\begin{description}
\sphinxlineitem{Args:}
\sphinxAtStartPar
X\_scaled(Dataframe): Matriz de datos escalados que seran la entrada al modelo.
y(Series): Vector de las variables de salida del modelo.

\end{description}

\end{fulllineitems}

\index{module@\spxentry{module}!entrenar\_modelos@\spxentry{entrenar\_modelos}}\index{entrenar\_modelos@\spxentry{entrenar\_modelos}!module@\spxentry{module}}\index{define\_model\_ann() (en el módulo entrenar\_modelos)@\spxentry{define\_model\_ann()}\spxextra{en el módulo entrenar\_modelos}}\phantomsection\label{\detokenize{Reduccion_Dimension:module-entrenar_modelos}}

\begin{fulllineitems}
\phantomsection\label{\detokenize{Reduccion_Dimension:entrenar_modelos.define_model_ann}}
\pysigstartsignatures
\pysiglinewithargsret
{\sphinxcode{\sphinxupquote{entrenar\_modelos.}}\sphinxbfcode{\sphinxupquote{define\_model\_ann}}}
{\sphinxparam{\DUrole{n}{caracteristicas}}\sphinxparamcomma \sphinxparam{\DUrole{n}{num\_clases}}}
{}
\pysigstopsignatures
\sphinxAtStartPar
Define un modelo de red neuronal con arquitectura ANN para clasificación multiclase.
\begin{description}
\sphinxlineitem{Args:}
\sphinxAtStartPar
caracteristicas (int): Número de características de entrada.
num\_clases (int): Número de clases para la clasificación.

\sphinxlineitem{Returns:}
\sphinxAtStartPar
model (Sequential): Modelo ANN compilado.

\end{description}

\end{fulllineitems}

\index{entrenar\_modelo\_ann() (en el módulo entrenar\_modelos)@\spxentry{entrenar\_modelo\_ann()}\spxextra{en el módulo entrenar\_modelos}}

\begin{fulllineitems}
\phantomsection\label{\detokenize{Reduccion_Dimension:entrenar_modelos.entrenar_modelo_ann}}
\pysigstartsignatures
\pysiglinewithargsret
{\sphinxcode{\sphinxupquote{entrenar\_modelos.}}\sphinxbfcode{\sphinxupquote{entrenar\_modelo\_ann}}}
{\sphinxparam{\DUrole{n}{X\_scaled}}\sphinxparamcomma \sphinxparam{\DUrole{n}{train\_idx}}\sphinxparamcomma \sphinxparam{\DUrole{n}{val\_idx}}\sphinxparamcomma \sphinxparam{\DUrole{n}{y}}\sphinxparamcomma \sphinxparam{\DUrole{n}{mode}}}
{}
\pysigstopsignatures
\sphinxAtStartPar
Función que entrena el modelo ANN utilizando diferentes métodos (secuencial, multihilo, multiproceso).
\begin{description}
\sphinxlineitem{Args:}
\sphinxAtStartPar
X\_scaled (array): Datos de entrada escalados.
train\_idx (array): Índices de entrenamiento.
val\_idx (array): Índices de validación.
y (array): Etiquetas de clase.
mode (str): Modo de entrenamiento (“secuencial”, “multihilo”, “multiproceso”, “n\_jobs”).

\sphinxlineitem{Returns:}
\sphinxAtStartPar
model (Sequential): Modelo entrenado.
x\_val (array): Datos de validación.
y\_val (array): Etiquetas de validación.
history (History): Historial del entrenamiento.

\end{description}

\end{fulllineitems}

\index{entrenar\_modelo\_knn() (en el módulo entrenar\_modelos)@\spxentry{entrenar\_modelo\_knn()}\spxextra{en el módulo entrenar\_modelos}}

\begin{fulllineitems}
\phantomsection\label{\detokenize{Reduccion_Dimension:entrenar_modelos.entrenar_modelo_knn}}
\pysigstartsignatures
\pysiglinewithargsret
{\sphinxcode{\sphinxupquote{entrenar\_modelos.}}\sphinxbfcode{\sphinxupquote{entrenar\_modelo\_knn}}}
{\sphinxparam{\DUrole{n}{X\_scaled}}\sphinxparamcomma \sphinxparam{\DUrole{n}{train\_idx}}\sphinxparamcomma \sphinxparam{\DUrole{n}{val\_idx}}\sphinxparamcomma \sphinxparam{\DUrole{n}{y}}\sphinxparamcomma \sphinxparam{\DUrole{n}{mode}}}
{}
\pysigstopsignatures
\sphinxAtStartPar
Función que entrena el modelo K\sphinxhyphen{}Nearest Neighbors utilizando diferentes métodos (secuencial, multihilo, multiproceso).
\begin{description}
\sphinxlineitem{Args:}
\sphinxAtStartPar
X\_scaled (array): Datos de entrada escalados.
train\_idx (array): Índices de entrenamiento.
val\_idx (array): Índices de validación.
y (array): Etiquetas de clase.
mode (str): Modo de entrenamiento (“secuencial”, “multihilo”, “multiproceso”, “n\_jobs”).

\sphinxlineitem{Returns:}
\sphinxAtStartPar
modelo\_fold (KNeighborsClassifier): Modelo entrenado.
x\_val (array): Datos de validación.
y\_val (array): Etiquetas de validación.

\end{description}

\end{fulllineitems}

\index{entrenar\_modelo\_rfc() (en el módulo entrenar\_modelos)@\spxentry{entrenar\_modelo\_rfc()}\spxextra{en el módulo entrenar\_modelos}}

\begin{fulllineitems}
\phantomsection\label{\detokenize{Reduccion_Dimension:entrenar_modelos.entrenar_modelo_rfc}}
\pysigstartsignatures
\pysiglinewithargsret
{\sphinxcode{\sphinxupquote{entrenar\_modelos.}}\sphinxbfcode{\sphinxupquote{entrenar\_modelo\_rfc}}}
{\sphinxparam{\DUrole{n}{X\_scaled}}\sphinxparamcomma \sphinxparam{\DUrole{n}{train\_idx}}\sphinxparamcomma \sphinxparam{\DUrole{n}{val\_idx}}\sphinxparamcomma \sphinxparam{\DUrole{n}{y}}\sphinxparamcomma \sphinxparam{\DUrole{n}{mode}}}
{}
\pysigstopsignatures
\sphinxAtStartPar
Función que entrena el modelo de Random Forest utilizando diferentes métodos (secuencial, multihilo, multiproceso).
\begin{description}
\sphinxlineitem{Args:}
\sphinxAtStartPar
X\_scaled (array): Datos de entrada escalados.
train\_idx (array): Índices de entrenamiento.
val\_idx (array): Índices de validación.
y (array): Etiquetas de clase.
mode (str): Modo de entrenamiento (“secuencial”, “multihilo”, “multiproceso”, “n\_jobs”).

\sphinxlineitem{Returns:}
\sphinxAtStartPar
modelo\_fold (RandomForestClassifier): Modelo entrenado.
x\_val (array): Datos de validación.
y\_val (array): Etiquetas de validación.

\end{description}

\end{fulllineitems}

\index{param\_KNN() (en el módulo entrenar\_modelos)@\spxentry{param\_KNN()}\spxextra{en el módulo entrenar\_modelos}}

\begin{fulllineitems}
\phantomsection\label{\detokenize{Reduccion_Dimension:entrenar_modelos.param_KNN}}
\pysigstartsignatures
\pysiglinewithargsret
{\sphinxcode{\sphinxupquote{entrenar\_modelos.}}\sphinxbfcode{\sphinxupquote{param\_KNN}}}
{}
{}
\pysigstopsignatures
\sphinxAtStartPar
Función que define los parámetros del modelo KNN.
\begin{description}
\sphinxlineitem{Returns:}
\sphinxAtStartPar
parametros\_KNN (dict): Parámetros del modelo KNN.

\end{description}

\end{fulllineitems}

\index{param\_RFC() (en el módulo entrenar\_modelos)@\spxentry{param\_RFC()}\spxextra{en el módulo entrenar\_modelos}}

\begin{fulllineitems}
\phantomsection\label{\detokenize{Reduccion_Dimension:entrenar_modelos.param_RFC}}
\pysigstartsignatures
\pysiglinewithargsret
{\sphinxcode{\sphinxupquote{entrenar\_modelos.}}\sphinxbfcode{\sphinxupquote{param\_RFC}}}
{}
{}
\pysigstopsignatures
\sphinxAtStartPar
Función que define los parámetros del modelo Random Forest.
\begin{description}
\sphinxlineitem{Returns:}
\sphinxAtStartPar
parametros\_RFC (dict): Parámetros del modelo Random Forest.

\end{description}

\end{fulllineitems}

\index{module@\spxentry{module}!Seleccion\_ejecucion@\spxentry{Seleccion\_ejecucion}}\index{Seleccion\_ejecucion@\spxentry{Seleccion\_ejecucion}!module@\spxentry{module}}\index{multihilo() (en el módulo Seleccion\_ejecucion)@\spxentry{multihilo()}\spxextra{en el módulo Seleccion\_ejecucion}}\phantomsection\label{\detokenize{Reduccion_Dimension:module-Seleccion_ejecucion}}

\begin{fulllineitems}
\phantomsection\label{\detokenize{Reduccion_Dimension:Seleccion_ejecucion.multihilo}}
\pysigstartsignatures
\pysiglinewithargsret
{\sphinxcode{\sphinxupquote{Seleccion\_ejecucion.}}\sphinxbfcode{\sphinxupquote{multihilo}}}
{\sphinxparam{\DUrole{n}{X\_scaled}}\sphinxparamcomma \sphinxparam{\DUrole{n}{y}}\sphinxparamcomma \sphinxparam{\DUrole{n}{kfold}}\sphinxparamcomma \sphinxparam{\DUrole{n}{metodo}}}
{}
\pysigstopsignatures
\sphinxAtStartPar
Entrena un modelo de clasificación secuencialmente.
Según el valor metodo llamará a la función de entrenamiento correspondiente.
Args:
\begin{quote}

\sphinxAtStartPar
X\_scaled (np.ndarray): Datos de entrada escalados.
y (np.ndarray): Etiquetas de los datos.
kfold (StratifiedKFold): Objeto de validación cruzada estratificada.
metodo (str): Método de entrenamiento a utilizar («RFC», «KNN», «ANN»).
\end{quote}

\end{fulllineitems}

\index{multiproceso() (en el módulo Seleccion\_ejecucion)@\spxentry{multiproceso()}\spxextra{en el módulo Seleccion\_ejecucion}}

\begin{fulllineitems}
\phantomsection\label{\detokenize{Reduccion_Dimension:Seleccion_ejecucion.multiproceso}}
\pysigstartsignatures
\pysiglinewithargsret
{\sphinxcode{\sphinxupquote{Seleccion\_ejecucion.}}\sphinxbfcode{\sphinxupquote{multiproceso}}}
{\sphinxparam{\DUrole{n}{X\_scaled}}\sphinxparamcomma \sphinxparam{\DUrole{n}{y}}\sphinxparamcomma \sphinxparam{\DUrole{n}{kfold}}\sphinxparamcomma \sphinxparam{\DUrole{n}{metodo}}}
{}
\pysigstopsignatures
\sphinxAtStartPar
Procede a realizar un entrenamiento secuencial del modelo.
Según el valor metodo llamará a la función de entrenamiento correspondiente.
\begin{description}
\sphinxlineitem{Args:}
\sphinxAtStartPar
X\_scaled (np.ndarray): Datos de entrada escalados.
y (np.ndarray): Etiquetas de los datos.
kfold (StratifiedKFold): Objeto de validación cruzada estratificada.
metodo (str): Método de entrenamiento a utilizar («RFC», «KNN», «ANN»).

\end{description}

\end{fulllineitems}

\index{n\_jobs() (en el módulo Seleccion\_ejecucion)@\spxentry{n\_jobs()}\spxextra{en el módulo Seleccion\_ejecucion}}

\begin{fulllineitems}
\phantomsection\label{\detokenize{Reduccion_Dimension:Seleccion_ejecucion.n_jobs}}
\pysigstartsignatures
\pysiglinewithargsret
{\sphinxcode{\sphinxupquote{Seleccion\_ejecucion.}}\sphinxbfcode{\sphinxupquote{n\_jobs}}}
{\sphinxparam{\DUrole{n}{X\_scaled}}\sphinxparamcomma \sphinxparam{\DUrole{n}{y}}\sphinxparamcomma \sphinxparam{\DUrole{n}{kfold}}\sphinxparamcomma \sphinxparam{\DUrole{n}{metodo}}}
{}
\pysigstopsignatures
\sphinxAtStartPar
Entrena un modelo de clasificación utilizando múltiples trabajos en paralelo.
Solo se utiliza para el método «RFC» o «KNN».
\begin{description}
\sphinxlineitem{Args:}
\sphinxAtStartPar
X\_scaled (np.ndarray): Datos de entrada escalados.
y (np.ndarray): Etiquetas de los datos.
kfold (StratifiedKFold): Objeto de validación cruzada estratificada.
metodo (str): Método de entrenamiento a utilizar («RFC», «KNN»).

\end{description}

\end{fulllineitems}

\index{single() (en el módulo Seleccion\_ejecucion)@\spxentry{single()}\spxextra{en el módulo Seleccion\_ejecucion}}

\begin{fulllineitems}
\phantomsection\label{\detokenize{Reduccion_Dimension:Seleccion_ejecucion.single}}
\pysigstartsignatures
\pysiglinewithargsret
{\sphinxcode{\sphinxupquote{Seleccion\_ejecucion.}}\sphinxbfcode{\sphinxupquote{single}}}
{\sphinxparam{\DUrole{n}{X\_scaled}}\sphinxparamcomma \sphinxparam{\DUrole{n}{y}}\sphinxparamcomma \sphinxparam{\DUrole{n}{kfold}}\sphinxparamcomma \sphinxparam{\DUrole{n}{metodo}}}
{}
\pysigstopsignatures
\sphinxAtStartPar
Realiza el entrenamiento del modelo de manera secuencial.
Según el valor metodo llamará a la función de entrenamiento correspondiente.
Args:
\begin{quote}

\sphinxAtStartPar
X\_scaled (np.ndarray): Datos de entrada escalados.
y (np.ndarray): Etiquetas de los datos.
kfold (StratifiedKFold): Objeto de validación cruzada estratificada.
metodo (str): Método de entrenamiento a utilizar («RFC», «KNN», «ANN»).
\end{quote}

\end{fulllineitems}


\sphinxstepscope


\chapter{Funciones Visualizacion}
\label{\detokenize{Funciones_Visualizacion:module-tsne}}\label{\detokenize{Funciones_Visualizacion:funciones-visualizacion}}\label{\detokenize{Funciones_Visualizacion::doc}}\index{module@\spxentry{module}!tsne@\spxentry{tsne}}\index{tsne@\spxentry{tsne}!module@\spxentry{module}}\index{tsne() (en el módulo tsne)@\spxentry{tsne()}\spxextra{en el módulo tsne}}

\begin{fulllineitems}
\phantomsection\label{\detokenize{Funciones_Visualizacion:tsne.tsne}}
\pysigstartsignatures
\pysiglinewithargsret
{\sphinxcode{\sphinxupquote{tsne.}}\sphinxbfcode{\sphinxupquote{tsne}}}
{\sphinxparam{\DUrole{n}{x}}\sphinxparamcomma \sphinxparam{\DUrole{n}{cluster\_labels}}\sphinxparamcomma \sphinxparam{\DUrole{n}{metodo}}}
{}
\pysigstopsignatures
\sphinxAtStartPar
Representa el conjunto de datos en 2D aplicando la reducción de dimensionalidad con t\sphinxhyphen{}SNE.
\begin{description}
\sphinxlineitem{Args:}
\sphinxAtStartPar
x (ndarray): Datos originales de entrada.
cluster\_labels (array\sphinxhyphen{}like): Etiquetas de cada dato.
metodo (str): Nombre del método de clustering utilizado (para el título del gráfico).

\sphinxlineitem{Returns:}
\sphinxAtStartPar
None: Muestra un gráfico con la representación t\sphinxhyphen{}SNE de los clústers.

\end{description}

\end{fulllineitems}

\index{module@\spxentry{module}!leer\_metricas@\spxentry{leer\_metricas}}\index{leer\_metricas@\spxentry{leer\_metricas}!module@\spxentry{module}}\index{get\_data() (en el módulo leer\_metricas)@\spxentry{get\_data()}\spxextra{en el módulo leer\_metricas}}\phantomsection\label{\detokenize{Funciones_Visualizacion:module-leer_metricas}}

\begin{fulllineitems}
\phantomsection\label{\detokenize{Funciones_Visualizacion:leer_metricas.get_data}}
\pysigstartsignatures
\pysiglinewithargsret
{\sphinxcode{\sphinxupquote{leer\_metricas.}}\sphinxbfcode{\sphinxupquote{get\_data}}}
{\sphinxparam{\DUrole{n}{excel\_data}}}
{}
\pysigstopsignatures
\sphinxAtStartPar
Extrae las métricas almacenadas en el diccionario generado por \sphinxtitleref{read\_excel\_file}, 
separándolas en distintos DataFrames.
\begin{quote}\begin{description}
\sphinxlineitem{Parámetros}
\sphinxAtStartPar
\sphinxstyleliteralstrong{\sphinxupquote{excel\_data}} (\sphinxstyleliteralemphasis{\sphinxupquote{dict}}\sphinxstyleliteralemphasis{\sphinxupquote{{[}}}\sphinxstyleliteralemphasis{\sphinxupquote{str}}\sphinxstyleliteralemphasis{\sphinxupquote{, }}\sphinxstyleliteralemphasis{\sphinxupquote{pandas.DataFrame}}\sphinxstyleliteralemphasis{\sphinxupquote{{]}}}) \textendash{} Diccionario con los datos del modelo.

\sphinxlineitem{Devuelve}
\sphinxAtStartPar
\begin{itemize}
\item {} 
\sphinxAtStartPar
\sphinxstylestrong{k\_metrics} (\sphinxstyleemphasis{pandas.DataFrame}): Métricas generales del modelo por fold.

\item {} 
\sphinxAtStartPar
\sphinxstylestrong{y\_test} (\sphinxstyleemphasis{pandas.DataFrame}): Clases reales de cada imagen.

\item {} 
\sphinxAtStartPar
\sphinxstylestrong{y\_pred} (\sphinxstyleemphasis{pandas.DataFrame}): Clases predichas por el modelo.

\item {} 
\sphinxAtStartPar
\sphinxstylestrong{histories} (\sphinxstyleemphasis{pandas.DataFrame}): Evolución del accuracy y pérdida (si está disponible).

\item {} 
\sphinxAtStartPar
\sphinxstylestrong{roc} (\sphinxstyleemphasis{pandas.DataFrame}): Datos para la curva ROC y AUC (si está disponible).

\end{itemize}


\sphinxlineitem{Tipo del valor devuelto}
\sphinxAtStartPar
tuple{[}pandas.DataFrame, pandas.DataFrame, pandas.DataFrame, pandas.DataFrame, pandas.DataFrame{]}

\end{description}\end{quote}

\end{fulllineitems}

\index{plot\_accuracy\_evolution() (en el módulo leer\_metricas)@\spxentry{plot\_accuracy\_evolution()}\spxextra{en el módulo leer\_metricas}}

\begin{fulllineitems}
\phantomsection\label{\detokenize{Funciones_Visualizacion:leer_metricas.plot_accuracy_evolution}}
\pysigstartsignatures
\pysiglinewithargsret
{\sphinxcode{\sphinxupquote{leer\_metricas.}}\sphinxbfcode{\sphinxupquote{plot\_accuracy\_evolution}}}
{\sphinxparam{\DUrole{n}{histories}}}
{}
\pysigstopsignatures
\sphinxAtStartPar
Muestra la evolución del accuracy en entrenamiento y validación a lo largo de las iteraciones.
\begin{quote}\begin{description}
\sphinxlineitem{Parámetros}
\sphinxAtStartPar
\sphinxstyleliteralstrong{\sphinxupquote{histories}} (\sphinxstyleliteralemphasis{\sphinxupquote{pandas.DataFrame}}) \textendash{} DataFrame con la evolución del accuracy y la pérdida 
en entrenamiento y validación.

\sphinxlineitem{Devuelve}
\sphinxAtStartPar
No retorna ningún valor, solo muestra la gráfica de evolución del accuracy.

\sphinxlineitem{Tipo del valor devuelto}
\sphinxAtStartPar
None

\end{description}\end{quote}

\end{fulllineitems}

\index{plot\_classes\_roc\_curves() (en el módulo leer\_metricas)@\spxentry{plot\_classes\_roc\_curves()}\spxextra{en el módulo leer\_metricas}}

\begin{fulllineitems}
\phantomsection\label{\detokenize{Funciones_Visualizacion:leer_metricas.plot_classes_roc_curves}}
\pysigstartsignatures
\pysiglinewithargsret
{\sphinxcode{\sphinxupquote{leer\_metricas.}}\sphinxbfcode{\sphinxupquote{plot\_classes\_roc\_curves}}}
{\sphinxparam{\DUrole{n}{roc}}}
{}
\pysigstopsignatures
\sphinxAtStartPar
Genera y muestra las curvas ROC individuales para cada clase.
\begin{quote}\begin{description}
\sphinxlineitem{Parámetros}
\sphinxAtStartPar
\sphinxstyleliteralstrong{\sphinxupquote{roc}} (\sphinxstyleliteralemphasis{\sphinxupquote{pandas.DataFrame}}) \textendash{} DataFrame con los datos de la curva ROC por clase.

\sphinxlineitem{Devuelve}
\sphinxAtStartPar
No retorna ningún valor, solo muestra las curvas ROC por clase.

\sphinxlineitem{Tipo del valor devuelto}
\sphinxAtStartPar
None

\end{description}\end{quote}

\end{fulllineitems}

\index{plot\_confusion\_matrix() (en el módulo leer\_metricas)@\spxentry{plot\_confusion\_matrix()}\spxextra{en el módulo leer\_metricas}}

\begin{fulllineitems}
\phantomsection\label{\detokenize{Funciones_Visualizacion:leer_metricas.plot_confusion_matrix}}
\pysigstartsignatures
\pysiglinewithargsret
{\sphinxcode{\sphinxupquote{leer\_metricas.}}\sphinxbfcode{\sphinxupquote{plot\_confusion\_matrix}}}
{\sphinxparam{\DUrole{n}{y\_test}}\sphinxparamcomma \sphinxparam{\DUrole{n}{y\_pred}}}
{}
\pysigstopsignatures
\sphinxAtStartPar
Genera y muestra una matriz de confusión para cada fold de validación.
\begin{quote}\begin{description}
\sphinxlineitem{Parámetros}\begin{itemize}
\item {} 
\sphinxAtStartPar
\sphinxstyleliteralstrong{\sphinxupquote{y\_test}} (\sphinxstyleliteralemphasis{\sphinxupquote{pandas.DataFrame}}) \textendash{} DataFrame con las clases reales.

\item {} 
\sphinxAtStartPar
\sphinxstyleliteralstrong{\sphinxupquote{y\_pred}} (\sphinxstyleliteralemphasis{\sphinxupquote{pandas.DataFrame}}) \textendash{} DataFrame con las clases predichas.

\end{itemize}

\sphinxlineitem{Devuelve}
\sphinxAtStartPar
No retorna ningún valor, solo muestra la matriz de confusión.

\sphinxlineitem{Tipo del valor devuelto}
\sphinxAtStartPar
None

\end{description}\end{quote}

\end{fulllineitems}

\index{plot\_metrics() (en el módulo leer\_metricas)@\spxentry{plot\_metrics()}\spxextra{en el módulo leer\_metricas}}

\begin{fulllineitems}
\phantomsection\label{\detokenize{Funciones_Visualizacion:leer_metricas.plot_metrics}}
\pysigstartsignatures
\pysiglinewithargsret
{\sphinxcode{\sphinxupquote{leer\_metricas.}}\sphinxbfcode{\sphinxupquote{plot\_metrics}}}
{\sphinxparam{\DUrole{n}{excel\_file}}}
{}
\pysigstopsignatures
\sphinxAtStartPar
Genera y muestra visualizaciones clave de las métricas obtenidas durante el entrenamiento de un modelo, incluyendo:
\begin{itemize}
\item {} 
\sphinxAtStartPar
Matriz de confusión para evaluar el desempeño del modelo en la clasificación.

\item {} 
\sphinxAtStartPar
Curvas ROC global y por clase para analizar la capacidad de discriminación del modelo.

\item {} 
\sphinxAtStartPar
Evolución del accuracy en entrenamiento y validación a lo largo de las iteraciones.

\end{itemize}
\begin{quote}\begin{description}
\sphinxlineitem{Parámetros}
\sphinxAtStartPar
\sphinxstyleliteralstrong{\sphinxupquote{excel\_file}} (\sphinxstyleliteralemphasis{\sphinxupquote{str}}) \textendash{} Ruta del archivo Excel que contiene los resultados del modelo.

\sphinxlineitem{Devuelve}
\sphinxAtStartPar
No retorna ningún valor, solo genera y muestra las visualizaciones correspondientes.

\sphinxlineitem{Tipo del valor devuelto}
\sphinxAtStartPar
None

\end{description}\end{quote}

\end{fulllineitems}

\index{plot\_roc\_curves() (en el módulo leer\_metricas)@\spxentry{plot\_roc\_curves()}\spxextra{en el módulo leer\_metricas}}

\begin{fulllineitems}
\phantomsection\label{\detokenize{Funciones_Visualizacion:leer_metricas.plot_roc_curves}}
\pysigstartsignatures
\pysiglinewithargsret
{\sphinxcode{\sphinxupquote{leer\_metricas.}}\sphinxbfcode{\sphinxupquote{plot\_roc\_curves}}}
{\sphinxparam{\DUrole{n}{roc}}}
{}
\pysigstopsignatures
\sphinxAtStartPar
Genera y muestra la curva ROC global junto con el área bajo la curva (AUC).
\begin{quote}\begin{description}
\sphinxlineitem{Parámetros}
\sphinxAtStartPar
\sphinxstyleliteralstrong{\sphinxupquote{roc}} (\sphinxstyleliteralemphasis{\sphinxupquote{pandas.DataFrame}}) \textendash{} DataFrame con los datos para la curva ROC global.

\sphinxlineitem{Devuelve}
\sphinxAtStartPar
No retorna ningún valor, solo muestra la curva ROC.

\sphinxlineitem{Tipo del valor devuelto}
\sphinxAtStartPar
None

\end{description}\end{quote}

\end{fulllineitems}

\index{read\_excel\_file() (en el módulo leer\_metricas)@\spxentry{read\_excel\_file()}\spxextra{en el módulo leer\_metricas}}

\begin{fulllineitems}
\phantomsection\label{\detokenize{Funciones_Visualizacion:leer_metricas.read_excel_file}}
\pysigstartsignatures
\pysiglinewithargsret
{\sphinxcode{\sphinxupquote{leer\_metricas.}}\sphinxbfcode{\sphinxupquote{read\_excel\_file}}}
{\sphinxparam{\DUrole{n}{excel\_file}}}
{}
\pysigstopsignatures
\sphinxAtStartPar
Lee un archivo Excel y devuelve su contenido en un diccionario, donde cada clave 
representa el nombre de una hoja y su valor correspondiente es un DataFrame con los datos de dicha hoja.
\begin{quote}\begin{description}
\sphinxlineitem{Parámetros}
\sphinxAtStartPar
\sphinxstyleliteralstrong{\sphinxupquote{excel\_file}} (\sphinxstyleliteralemphasis{\sphinxupquote{str}}) \textendash{} Ruta del archivo Excel a leer.

\sphinxlineitem{Devuelve}
\sphinxAtStartPar
Diccionario con los datos de cada hoja en formato DataFrame.

\sphinxlineitem{Tipo del valor devuelto}
\sphinxAtStartPar
dict{[}str, pandas.DataFrame{]}

\end{description}\end{quote}

\end{fulllineitems}

\index{module@\spxentry{module}!Matriz\_correlacion@\spxentry{Matriz\_correlacion}}\index{Matriz\_correlacion@\spxentry{Matriz\_correlacion}!module@\spxentry{module}}\index{Variables\_correlacion() (en el módulo Matriz\_correlacion)@\spxentry{Variables\_correlacion()}\spxextra{en el módulo Matriz\_correlacion}}\phantomsection\label{\detokenize{Funciones_Visualizacion:module-Matriz_correlacion}}

\begin{fulllineitems}
\phantomsection\label{\detokenize{Funciones_Visualizacion:Matriz_correlacion.Variables_correlacion}}
\pysigstartsignatures
\pysiglinewithargsret
{\sphinxcode{\sphinxupquote{Matriz\_correlacion.}}\sphinxbfcode{\sphinxupquote{Variables\_correlacion}}}
{\sphinxparam{\DUrole{n}{X}}\sphinxparamcomma \sphinxparam{\DUrole{n}{x\_pca}}\sphinxparamcomma \sphinxparam{\DUrole{n}{y}}}
{}
\pysigstopsignatures
\sphinxAtStartPar
Función que calcula la matriz de correlación entre las variables originales y la variable de salida,

\sphinxAtStartPar
y entre las variables transformadas por PCA y la variable de salida.
Se muestran las matrices de correlación en dos gráficos diferentes.
\begin{description}
\sphinxlineitem{Args:}
\sphinxAtStartPar
X (Dataframe): Datos no normalizados.
x\_pca (ndarray): Matriz de características transformadas por PCA.
y(Series): Vector de las variables de salida del modelo.

\end{description}

\end{fulllineitems}

\index{module@\spxentry{module}!Principales\_caracteristicas@\spxentry{Principales\_caracteristicas}}\index{Principales\_caracteristicas@\spxentry{Principales\_caracteristicas}!module@\spxentry{module}}\index{pca() (en el módulo Principales\_caracteristicas)@\spxentry{pca()}\spxextra{en el módulo Principales\_caracteristicas}}\phantomsection\label{\detokenize{Funciones_Visualizacion:module-Principales_caracteristicas}}

\begin{fulllineitems}
\phantomsection\label{\detokenize{Funciones_Visualizacion:Principales_caracteristicas.pca}}
\pysigstartsignatures
\pysiglinewithargsret
{\sphinxcode{\sphinxupquote{Principales\_caracteristicas.}}\sphinxbfcode{\sphinxupquote{pca}}}
{\sphinxparam{\DUrole{n}{x\_scaled}}}
{}
\pysigstopsignatures
\sphinxAtStartPar
Realiza PCA sobre los datos escalados y visualiza la varianza explicada por cada componente principal.

\sphinxAtStartPar
Args:
x\_scaled (DataFrame): Datos escalados.
\begin{description}
\sphinxlineitem{Returns:}
\sphinxAtStartPar
x\_pca (ndarray): Datos que se enviarán al modelo.

\end{description}

\end{fulllineitems}



\chapter{Indices y tablas}
\label{\detokenize{index:indices-y-tablas}}\begin{itemize}
\item {} 
\sphinxAtStartPar
\DUrole{xref}{\DUrole{std}{\DUrole{std-ref}{genindex}}}

\item {} 
\sphinxAtStartPar
\DUrole{xref}{\DUrole{std}{\DUrole{std-ref}{modindex}}}

\item {} 
\sphinxAtStartPar
\DUrole{xref}{\DUrole{std}{\DUrole{std-ref}{search}}}

\end{itemize}


\renewcommand{\indexname}{Índice de Módulos Python}
\begin{sphinxtheindex}
\let\bigletter\sphinxstyleindexlettergroup
\bigletter{c}
\item\relax\sphinxstyleindexentry{clustering}\sphinxstyleindexpageref{Clustering:\detokenize{module-clustering}}
\item\relax\sphinxstyleindexentry{comparacion\_n\_covariables}\sphinxstyleindexpageref{Funciones_Generales:\detokenize{module-comparacion_n_covariables}}
\indexspace
\bigletter{e}
\item\relax\sphinxstyleindexentry{entrenar\_modelos}\sphinxstyleindexpageref{Reduccion_Dimension:\detokenize{module-entrenar_modelos}}
\indexspace
\bigletter{g}
\item\relax\sphinxstyleindexentry{guardar\_metricas}\sphinxstyleindexpageref{Funciones_Generales:\detokenize{module-guardar_metricas}}
\indexspace
\bigletter{l}
\item\relax\sphinxstyleindexentry{leer\_metricas}\sphinxstyleindexpageref{Funciones_Visualizacion:\detokenize{module-leer_metricas}}
\indexspace
\bigletter{m}
\item\relax\sphinxstyleindexentry{Matriz\_correlacion}\sphinxstyleindexpageref{Funciones_Visualizacion:\detokenize{module-Matriz_correlacion}}
\item\relax\sphinxstyleindexentry{metricas\_clustering}\sphinxstyleindexpageref{Funciones_Generales:\detokenize{module-metricas_clustering}}
\item\relax\sphinxstyleindexentry{Modelos}\sphinxstyleindexpageref{Reduccion_Dimension:\detokenize{module-Modelos}}
\indexspace
\bigletter{n}
\item\relax\sphinxstyleindexentry{normalizar}\sphinxstyleindexpageref{Funciones_Generales:\detokenize{module-normalizar}}
\indexspace
\bigletter{p}
\item\relax\sphinxstyleindexentry{Principales\_caracteristicas}\sphinxstyleindexpageref{Funciones_Visualizacion:\detokenize{module-Principales_caracteristicas}}
\indexspace
\bigletter{s}
\item\relax\sphinxstyleindexentry{Seleccion\_ejecucion}\sphinxstyleindexpageref{Reduccion_Dimension:\detokenize{module-Seleccion_ejecucion}}
\item\relax\sphinxstyleindexentry{subir\_datos}\sphinxstyleindexpageref{Funciones_Generales:\detokenize{module-subir_datos}}
\indexspace
\bigletter{t}
\item\relax\sphinxstyleindexentry{tsne}\sphinxstyleindexpageref{Funciones_Visualizacion:\detokenize{module-tsne}}
\end{sphinxtheindex}

\renewcommand{\indexname}{Índice}
\printindex
\end{document}